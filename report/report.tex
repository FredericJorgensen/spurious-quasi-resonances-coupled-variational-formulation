\documentclass[10pt,journal,compsoc, onecolumn]{IEEEtran}

\usepackage[pdftex]{graphicx}   
\usepackage{cite}
\usepackage[utf8]{inputenc}
\usepackage[T1]{fontenc}
\usepackage{amsmath}
\usepackage{amsfonts}
\usepackage{amssymb}
\usepackage{hyperref}
\hypersetup{colorlinks=true, linkcolor=[rgb]{0,0,1}, citecolor=[rgb]{0,0,1}}
\usepackage{graphicx}
\usepackage{lmodern}
\usepackage{physics}
\usepackage[left=1cm,right=1cm,top=2cm,bottom=1.5cm]{geometry}
\usepackage{siunitx}
\usepackage{fancyhdr}
\usepackage{enumerate}
\usepackage{mhchem}
\usepackage{mathrsfs}
\usepackage{mathtools}
\usepackage{graphicx}
\usepackage{float}
\usepackage{xcolor}
\usepackage{mdframed}
\usepackage{csquotes}
\usepackage{trfsigns}
\usepackage{capt-of}
\usepackage{booktabs}
\usepackage{comment}
\usepackage{amsmath}
\usepackage{algorithm}
\usepackage{amsthm}
\usepackage[noend]{algpseudocode}
\usepackage[
singlelinecheck=false % <-- important
]{caption}

\makeatletter
\def\BState{\State\hskip-\ALG@thistlm}
\makeatother

\usepackage{fixltx2e}
\usepackage{xcolor}
\def\SPSB#1#2{\rlap{\textsuperscript{\textcolor{red}{#1}}}\SB{#2}}
\def\SP#1{\textsuperscript{\textcolor{red}{#1}}}
\def\SB#1{\textsubscript{\textcolor{blue}{#1}}}

\newcommand{\subparagraph}{}
\usepackage{titlesec}

\setcounter{secnumdepth}{4}

\titleformat{\paragraph}
{\normalfont\normalsize\bfseries}{\theparagraph}{1em}{}
\titlespacing*{\paragraph}
{0pt}{3.25ex plus 1ex minus .2ex}{1.5ex plus .2ex}

\mdfdefinestyle{exercise}{
	backgroundcolor=black!10,roundcorner=8pt,hidealllines=true,nobreak
}



\newtheorem{theorem}{Theorem}[section]
\newtheorem{corollary}{corollary}[theorem]
\newtheorem{lemma}[theorem]{Lemma}
\newtheorem{definition}[theorem]{Definition}
\newtheorem{proposition}[theorem]{Proposition}
\newtheorem{remark}[theorem]{Remark}
\newtheorem{example}[theorem]{Example}






\begin{document}

\title{Spurious Quasi-Resonances for Stabilized BIE-Volume Formulations for Helmholtz Transition Problem}

\author{Frederic Jørgensen. \textit{\today}}




\IEEEtitleabstractindextext{%
\begin{abstract}
    tbd 
Lorem ipsum dolor sit amet, consetetur sadipscing elitr, sed diam nonumy eirmod tempor invidunt ut labore et dolore magna aliquyam erat, sed diam voluptua. At vero eos et accusam et justo duo dolores et ea rebum. Stet clita kasd gubergren, no sea takimata sanctus est Lorem ipsum dolor sit amet. Lorem ipsum dolor sit amet, consetetur sadipscing elitr, sed diam nonumy eirmod tempor invidunt ut labore et dolore magna aliquyam erat, sed diam voluptua. At vero eos et accusam et justo duo dolores et ea rebum. Stet clita kasd gubergren, no sea takimata sanctus est Lorem ipsum dolor sit amet.
\end{abstract}
}



\maketitle

\IEEEdisplaynontitleabstractindextext
\IEEEpeerreviewmaketitle



\section{Introduction}
Let \(\Omega^- \subset \mathbb{R}^d, d > 0\) be a bounded Lipschitz domain and 
define \(\Omega^{+}:=\mathbb{R}^{d} \backslash \overline{\Omega^{-}}\), 
\(\Gamma := \partial \Omega^-\).
For any function $f$ on $\mathbb{R}^d$ define $f^\pm :=f|_{\Omega^{\pm}}$. 
Consider the Neumann and Dirichlet trace operators 
\begin{align}
    \gamma_{D}^{\pm}: & H_{\mathrm{loc}}^{1}\left(\Omega^{\pm}\right) \rightarrow H^{\frac{1}{2}}(\Gamma), \left(\gamma_{D}^{\pm} f\right)({x}):=f({x}) \nonumber \\
    \gamma_{N}^{\pm}: & H_{\mathrm{loc}}\left(\Delta, \Omega^{\pm}\right) \rightarrow H^{-\frac{1}{2}}(\Gamma), \left(\gamma_{N}^{\pm} f\right)({x}):=\nabla f({x}) \cdot \mathbf{n}({x}) \nonumber
\end{align}
where \(H_{\mathrm{loc}}^{1}\left(\Omega^{\pm}\right)\),
\(H_{\mathrm{loc}}\left(\Delta, \Omega^{\pm}\right)\), 
\(H^{\frac{1}{2}}(\Gamma)\), and \(H^{-\frac{1}{2}}(\Gamma)\)
are defined in chapter 3 of  \cite{mclean2000strongly}.
Now, the Cauchy trace is \(\gamma_{C}^{\pm}: H_{\mathrm{loc}}^{1}\left(\Omega^{\pm}, \Delta\right) \rightarrow H^{1 / 2}(\Gamma) \times H^{-1 / 2}(\Gamma)\) with values given 
by \(\gamma_{C}^{\pm}:=\left(\gamma_{D}^{\pm}, \gamma_{N}^{\pm}\right)\). 
This concludes all required definitions to formulate the Helmholtz transmission problem.

\begin{definition}[Helmholtz transmission problem]
For $\tilde \kappa, c_i, c_o > 0$ and \(\mathbf{f} \in H^{1 / 2}(\Gamma) \times H^{-1 / 2}(\Gamma)\) find \(u \in H_{\operatorname{loc}}^{1}\left(\mathbb{R}^{d} \backslash \Gamma\right)\) 
such that

\begin{align}
    \left(\Delta+\tilde\kappa^{2} c_{i}\right) U^{-} &=0  \quad \text { in } \Omega^{-}  \nonumber \\
    \left(\Delta+\tilde\kappa^{2} c_{o}\right) U^{+} &=0  \quad \text { in } \Omega^{+} \label{eq:helmholtz} \\
    \gamma_{C}^{+} U^{+} - \gamma_{C}^{-} U^{-} &=\mathbf{f} \quad  \text { on } \Gamma. \nonumber 
\end{align}
Additionally, $u$ must satisfy the Sommerfeld radiation condition 
\(\lim\limits_{r \rightarrow \infty} r^{\frac{d-1}{2}}\left(\frac{\partial U(x)}{\partial r}-i \sqrt{c_o} \tilde \kappa U(x)\right)=0\) 
where $r$ refers to the radial spherical coordinate. 
\end{definition} 
This problem is well-posed and the solution is unique as shown in Lemma 2.2 in \cite{moiola2019acoustic}.\\
Hiptmair et al. considered the \textit{single-trace formulations} (STF) of this problem  \cite{hiptmair2021spurious}. 
This is a reformulation of this problem in terms of \textit{boundary integral equations} (BIEs). 
When investigating the case $c_i < c_o$ they found that the involved  \textit{boundary integral operators} (BIOs) as a
 function of $\tilde \kappa$ exposed a nonphysical resonanace behavior. 
More specifically, the operator norm of the STF BIOs was found to have resonances (called \textit{spurious quasi-resonances})
that the norm of the solution operator did not.
Formally, Hiptmair et al. defined the solution operator as follows \cite{hiptmair2021spurious}.
\begin{definition}
    Given positive real numbers \(k, c_{i}\), and \(c_{o}\), 
    $S: H^{\frac{1}{2}}(\Gamma) \times H^{-\frac{1}{2}}(\Gamma) \rightarrow  H^{\frac{1}{2}}(\Gamma) \times H^{-\frac{1}{2}}(\Gamma)$ is the solution operator if 
     \(S\left(c_{i}, c_{o}\right) \mathbf{f}:=\gamma_{C}^{-} u\)
    where \(u\) solves eq \ref{eq:helmholtz}.
\end{definition}
Hiptmair et al. were able to remove these \textit{spurious quasi-resonances} 
using an augmented formulation of the BIEs.  \\
The goal of this paper is to investigate the occurence of \textit{spurious quasi-resonance}
in the  \textit{regularized variational formulation} of the Helmholtz transmission problem 
as proposed in P. Meury's doctoral thesis \cite{meury2007stable}. 
For easier notation let $\kappa = \tilde \kappa \sqrt{c_o}$ on the following pages.
\newpage
\begin{definition}[Regularized variational formulation]
    Find \(U \in H^{1}(\Omega), \theta \in H^{-1 / 2}(\Gamma)\) and \(p \in H^{1}(\Gamma)\) such that for all \(V \in H^{1}(\Omega), \varphi \in H^{-1 / 2}(\Gamma)\)
    and \(q \in H^{1}(\Gamma)\) there holds

    \begin{align}
        \mathrm{q}_{\kappa}(U, V)+\left(\mathrm{W}_{\kappa}\left(\gamma_{D}^{-} U\right), \gamma_{D}^{-} V\right)_{\Gamma}-\left((\frac{1}{2} \mathrm{ld}-\mathrm{K}_{\kappa}^{\prime})(\theta), \gamma_{D}^{-} V\right)_{\Gamma} &=g_1(V) \nonumber\\
        \left((\frac{1}{2} \mathrm{ld}-\mathrm{K}_{\kappa})\left(\gamma_{D}^{-} U\right), \varphi\right)_{\Gamma}+\left(\mathrm{V}_{\kappa}(\theta), \varphi\right)_{\Gamma}+i \overline{\eta}(p, \varphi)_{\Gamma} &=\overline{g_2(\varphi)} \label{eq:variational_formulation}\\
        -\left(\mathrm{W}_{\kappa}\left(\gamma_{D}^{-} U\right), q\right)_{\Gamma}-\left((\mathrm{K}_{\kappa}^{\prime}+\frac{1}{2} \mathrm{Id})(\theta), q\right)_{\Gamma}+\mathrm{b}(p, q) &=g_3(q). \nonumber
    \end{align}

    where we have 
    $$
    \begin{aligned} 
        g_1(V) &:=(f, V)_{\Omega}-\left(g_{N}, \gamma_{D}^{-} V\right)_{\Gamma}-\left(\mathrm{W}_{\kappa}\left(g_{D}\right), \gamma_{D}^{-} V\right)_{\Gamma} \\ 
        g_2(V)(\varphi) &:=\left(\varphi,\left(\mathrm{K}_{\kappa}-\frac{1}{2} \mathrm{ld}\right)\left(g_{D}\right)\right)_{\Gamma} \\ 
        g_3(V)(q) &:=\left(\mathrm{W}_{\kappa}\left(g_{D}\right), q\right)_{\Gamma} \\ 
        \mathrm{q}_{\kappa}(U, V)& :=\int_{\Omega} \operatorname{grad} U \cdot \operatorname{grad} \bar{V}-\kappa^{2} n(\mathbf{x}) U \bar{V} \mathrm{~d} \mathbf{x} \\
        \mathrm{b}(p, q)& :=\left(\operatorname{grad}_{\Gamma} p, \operatorname{grad}_{\Gamma} q\right)_{\Gamma}+(p, q)_{\Gamma}.
    \end{aligned}
    $$

\end{definition}
This formulation is derived from the Helmholtz transmission problem by partially integrating the Helmholtz equation, 
applying Green's first formula, coupling the resulting variational problem to 
the BIEs using Dirichlet-to-Neumann maps, and transforming the Cauchy trace \cite{meury2007stable} 
\footnote{The formulation here is equivalent to the case 
 $U_i = 0, f = 0, n(x) = c_i / c_o$ in section 3 of \cite{meury2007stable}.}.
\begin{remark}
    From here on, we will use the constant $\tilde c := \frac{c_i}{c_o}$. This reduces the amount of parameters 
    to consider. Moreover,
    it makes more clear that the resonance behavior 
    depends on the ratio and not the scaling
    of the refractive indices.
\end{remark}
To investigate the occurence of spurious quasi-resonances, 
we consider the simple example where $d = 2$ and $\Omega^- = B_1(0)$ in the following chapters if not mentioned otherwise. 
\section{Constructing the Galerkin Matrix}
A numerical solution of the regularized variational formulation requires the 
construction of the Galerkin matrix of eq. \ref{eq:variational_formulation}. 
To proceed with this, we must first construct a
 basis of a subspace of the the solution space $H^{1}(\Omega)\times H^{-\frac{1}{2}}(\Gamma) \times H^1(\Gamma)$. 
Orthonormality is required since the representation matrix 
of an endomorphism is orthogonal if and only if the representation basis is orthogonal. 
As our goal is to approximate operator norms discretely, we must choose a orthonormal basis, 
allowing us to use an SVD for this\footnote{How we will do this will be defined more specifically on the following pages.}.\\
Before we can derive the Galerkin matrix, we have to define an orthonormal basis of a finite subspace $H^{1}(\Omega)\times H^{-\frac{1}{2}}(\Gamma) \times H^1(\Gamma)$, 
so that we can formulate the problem in terms of finding the coefficients with respect to this basis. \\ 
To simplify the following derivations, we will consider the following space 
\begin{definition} 
    For \(0 \leq s<\infty\) the space \(\mathcal{H}^{s}_{\tilde \kappa}(X)\) is defined as the subspace of all functions $\varphi \in L^{2}(X)$
    such that
    $$
    \sum_{n \in \mathbb{Z}}\left(\tilde \kappa^2+n^{2}\right)^{s}\left|\varphi_{n}\right|^{2}<\infty. 
    $$
    for the Fourier coefficients \(\varphi_{n}\) of \(\varphi\). We define an inner product on this space by
    $$
    (\varphi, \psi)_{\mathcal{H}^{s}(\mathbb{S})}:=\sum_{n \in \mathbb{Z}}\left(\tilde \kappa^2+n^{2}\right)^{s} \varphi_{n} \overline{\psi_{n}}.
    $$
\end{definition}
A justification for this definition is given in Meury (lemma 3.22) and the norm is introduced in (2021 Spurious resonances).
\\
We note that, within the variational formulation, $U$ is only evaluated on the boundary $\Gamma$\footnote{This is actually not clear for the term $\mathrm{q}_{\kappa}$. However, we will prove this assumption later.}, 
so that we can restrict $U$ to $H^{\frac{1}{2}}(\Gamma)$, which simplifies our problem. \\
Now we can can define a complete orthogonal system $\mathcal(H)^{\frac{1}{2}}(\Gamma) \times \mathcal(H)^{-\frac{1}{2}}(\Gamma) \times \mathcal(H)^{1}(\Gamma)$. 
We choose $U_{n}=v_{n} V_{n}(1, \phi)$, $\theta_{n}=w_{n} e^{i n \phi}$, $p_{n}=l_{n} e^{i n \phi}$ 
of \(\mathcal{H}^{-\frac{1}{2}}(\Gamma)\) and \(\mathcal{H}^{1}(\Gamma)\) respectively, where
\begin{align} 
    v_{n}=\frac{1}{\sqrt{2 \pi\left(\tilde \kappa^2+n^{2}\right)}\left|J_{n}\left(\sqrt{\frac{c_{i}}{c_{o}}} \kappa\right)\right|}, V_{n}=J_{n}\left(\sqrt{\frac{c_{i}}{c_{o}}} \kappa r\right) e^{i n \phi} \nonumber \\
    w_{n}=\frac{\left(\tilde \kappa^2+n^{2}\right)^{\frac{1}{4}}}{\sqrt{2 \pi}} \label{eq:coefficients}\\
    l_{n}=\frac{1}{\sqrt{2 \pi\left(\tilde \kappa^2+n^{2}\right)}} \nonumber
\end{align}
CITE MEURY 





\begin{comment}
\subsubsection*{Notes}
In the calculation above we used \(\nabla f = \frac{\partial f}{\partial r} \hat{\mathbf{r}}+\frac{1}{r} \frac{\partial f}{\partial \theta} \hat{\boldsymbol{\theta}}\) and eliminated the $r-$ dependency.

\textit{Note (from NumPDE Advanced):} 1. The Dirichlet trace space \(H^{\frac{1}{2}}(\Gamma)\) is the Hilbert space obtained by completion of \(\left.C^{\infty}(\bar{\Omega})\right|_{\Gamma}\) with
respect to the energy norm
$$
\|\mathfrak{u}\|_{H^{\frac{1}{2}}(\Gamma)}:=\inf \left\{\|v\|_{H^{1}(\Omega)}: v \in C^{\infty}(\bar{\Omega}), T_{D} v=\mathfrak{u}\right\},\left.\quad \mathfrak{u} \in C^{\infty}(\bar{\Omega})\right|_{\Gamma}
$$. 
\end{comment}

%2. The Dirichlet trace \(\mathrm{T}_{D}\) according to Def. 1.3.3 can be extended to a continuous and surjective linear
%$operator \(\mathrm{T}_{D}: H^{1}(\Omega) \rightarrow H^{\frac{1}{2}}(\Gamma)\)
\section{Constructing the Galerkin Matrix}
Now let's restrict ourselves to the space $\mathcal{S}_N^{\frac{1}{2}} 
\times \mathcal{S}_N^{-\frac{1}{2}} \times \mathcal{S}_N^{1}$ where $N \in \mathbb{N}$ 
and $\mathcal{S}_N^{\frac{1}{2}}$ is the restriction to $\mathrm{span}(U_{-N}, -U_{-N+1}, ..., U_N)$ and similarly 
for $\mathcal{S}_N^{-\frac{1}{2}}$ and $\mathcal{S}_N^{1}$ and construct the Galerkin Matrix.
\subsection{Redefining the problem}
We reduce our problem to the space \(\mathcal{S}^{\frac{1}{2}} \times \mathcal{S}^{-\frac{1}{2}} \times \mathcal{S}^{1}\):
\begin{definition}[The restricted problem]
    \label{def:restricted_variational_problem}
    Find \((u, \theta, p) \in  \mathcal{S}_N^{\frac{1}{2}} \times \mathcal{S}_N^{-\frac{1}{2}} \times \mathcal{S}_N^{1}\) 
    such that for all $(V, \varphi, p) \in \mathcal{S}_N^{\frac{1}{2}} \times \mathcal{S}_N^{-\frac{1}{2}} \times \mathcal{S}_N^{1}$ there holds
    $$
    \begin{aligned}
        \mathrm{q}_{\kappa}(U, V)+\left(\mathrm{W}_{\kappa}\left(\gamma_{D}^{-} U\right), \gamma_{D}^{-} V\right)_{\Gamma}-\left((\frac{1}{2} \mathrm{ld}-\mathrm{K}_{\kappa}^{\prime})(\theta), \gamma_{D}^{-} V\right)_{\Gamma} &=g_1(V) \\
        \left((\frac{1}{2} \mathrm{ld}-\mathrm{K}_{\kappa})\left(\gamma_{D}^{-} U\right), \varphi\right)_{\Gamma}+\left(\mathrm{V}_{\kappa}(\theta), \varphi\right)_{\Gamma}+i \overline{\eta}(p, \varphi)_{\Gamma} &=\overline{g_2(V)}(\varphi) \\
        -\left(\mathrm{W}_{\kappa}\left(\gamma_{D}^{-} U\right), q\right)_{\Gamma}-\left((\mathrm{K}_{\kappa}^{\prime}+\frac{1}{2} \mathrm{Id})(\theta), q\right)_{\Gamma}+\mathrm{b}(p, q) &=g_3(V)(q).
    \end{aligned}
    $$
\end{definition}
Now let's simplify the notation of this problem a bit. First, let's extend 
$(u, \theta, p) = \sum\limits_{n = -N}^N (C^U_n U_n, C^\theta_n \theta_n, C^p_n p_n)$, 
$f_1= \sum\limits_{j = -N}^N \frac{1}{2\pi}f_1^j e^{i j \theta}$ and $f_2= \sum\limits_{j = -N}^N \frac{1}{2\pi} f_2^j e^{i j \theta}$.
\\Define $\vec{C}_n = (C^U_n, C^\theta_n, C^p_n)$.
\\Let's introduce a few new constants before deriving the Galerkin Matrix. 

\begin{definition}
    Define the notation $P^{ab}_n = (a_n, b_n)_\Gamma$ where $a_n, b_n \in L^2(\Gamma)$. 
    In particular, the following values will be helpful: 
    \begin{itemize}
        \item $P^{UU}_n = 2\pi |\tilde v_n|^2$
        \item $P^{\theta U}_n = 2\pi w_n \overline{\tilde v_n}$
        \item $P^{U\theta}_n = \overline{P^{\theta U}_n}$
        \item $P^{\theta \theta}_n = 2 \pi |w_n|^2$
        \item $P^{p \theta}_n = 2\pi l_n\overline{w_n}$
        \item $P^{Up}_n = 2\pi \tilde v_n \overline{l_n}$
        \item $P^{\theta p}_n = \overline{P^{p\theta}_n }$
    \end{itemize}
    where $\tilde v_n = J_n(\sqrt{\tilde c}\kappa )v_n$.
\end{definition}
Let's introduce the following matrix and vector: 
\begin{definition}
    \label{def:galerkin_matrix}
    $$
    A_n:= 
    \begin{pmatrix}
        (\alpha_n + \lambda_n^{(w)} P^{UU}_n)  & - (\frac{1}{2} - \lambda_n^{(K')}) P_n^{\theta U} & 0\\
        (\frac{1}{2} - \lambda_n^{(K)})P^{U\theta}_n  & \lambda^{(V)}_n P^{\theta \theta}  &  i \overline{\eta}     P^{p \theta} \\
        -\lambda_n^{(W)} P^{Up}_n  &  - (\lambda^{(K')}_n + \frac{1}{2})  P^{\theta p} & \beta_n 
    \end{pmatrix}
    $$
    and 
    $$
    \vec{b}_n = 
    \begin{bmatrix}
       g_1(U_n)\\
        \overline{g_2}(\theta_n) \\
        g_3(p_n)
    \end{bmatrix}
    = 
    \begin{bmatrix}
        - \overline{\tilde{v_n}} f_2^n - \lambda_n^{(W)} \overline{\tilde{v_n}} f_1^n \\
        (\lambda_n^{(K)} - 0.5)  \overline{w_n} f_1^n\\
        \lambda_n^{(W)} \overline{l_n} f_1^n
    \end{bmatrix}
= f_1^n 
\underbrace{
\begin{bmatrix}
- \lambda_n^{(W)} \overline{\tilde{v_n}} \\
(\lambda_n^{(K)} - 0.5) \overline{w_n}  \\
\lambda_n^{(W)} \overline{l_n}
\end{bmatrix}}_{\vec{x_1}}
+ f_2^n
\underbrace{\begin{bmatrix}
    - \overline{\tilde{v_n}} \\
    0 \\
    0
\end{bmatrix}}_{\vec{x_1}}
    $$
    where $\alpha_n, \beta_n$ are constants that are defined below. 
\end{definition}
The following theorem summarises the Galerkin formulation of the problem: 
\begin{theorem}
    \label{def:galerkin_matrix}
    \((u, \theta, p) \in  \mathcal{H}^{\frac{1}{2}} \times \mathcal{H}^{-\frac{1}{2}} \times \mathcal{H}^{1}\) solves the problem in definition \ref{def:restricted_variational_problem}.
    if and only if $A_n\vec{C_n} = \vec{b}_n$ for all  $n = -N, -N + 1, ..., N$. 
\end{theorem}

Before we proof this we need the following Lemma from P. Meury doctoral thesis: 
\begin{lemma}
    The following eigenvalue equations hold for $y_n = e^{i n \phi}$:
    $$
    \begin{aligned} 
        \mathrm{V}_{\kappa}\left(y_{n}\right) &=\lambda_{n}^{(\mathrm{V})} y_{n}, & \lambda_{n}^{(\mathrm{V})} &:=\frac{i \pi}{2} J_{n}(\kappa) H_{n}^{(1)}(\kappa) \\ 
        \mathrm{K}_{\kappa}\left(y_{n}\right) &=\lambda_{n}^{(\mathrm{K})} y_{n}, & \lambda_{n}^{(\mathrm{K})} &:=\frac{i \pi \kappa}{2} J_{n}(\kappa) H_{n}^{(1)^{\prime}}(\kappa)+\frac{1}{2}=\frac{i \pi \kappa}{2} J_{n}^{\prime}(\kappa) H_{n}^{(1)}(\kappa)-\frac{1}{2} \\ 
        \mathrm{~K}_{\kappa}^{\prime}\left(y_{n}\right) &=\lambda_{n}^{\left(\mathrm{K}^{\prime}\right)} y_{n}, & \lambda_{n}^{\left(\mathrm{K}^{\prime}\right)} &:=\frac{i \pi \kappa}{2} J_{n}(\kappa) H_{n}^{(1)^{\prime}}(\kappa)+\frac{1}{2}=\frac{i \pi \kappa}{2} J_{n}^{\prime}(\kappa) H_{n}^{(1)}(\kappa)-\frac{1}{2} \\ 
        \mathcal{W}_{\kappa}\left(y_{n}\right) &=\lambda_{n}^{(\mathrm{W})} y_{n}, & \lambda_{n}^{(\mathrm{W})} &:= - \frac{i \pi \kappa^{2}}{2} J_{n}^{\prime}(\kappa) H_{n}^{(1)^{\prime}}(\kappa).
    \end{aligned}
    $$
    in particular these eigenvalue equations are satisfied for $U_n, \theta_n, p_n$.
\end{lemma}
The following formulas will also be helpful for the proof.
\begin{lemma}
    $q_\kappa(U_n, U_m) = 2 \pi  \delta_{nm}|v_n|^2\kappa  J_n(\kappa ) J_n'(\kappa ) =: \alpha_n$.
\end{lemma}


\begin{proof}[Proof of Lemma]
    We start by using Greens first formula 
    \(\int\limits_{U}(\psi \Delta \varphi+\nabla \psi \cdot \nabla \varphi) d V=\oint_{\partial U} \psi \nabla \varphi \cdot d \mathbf{S}\)
    to see that $$q_\kappa(U_n, U_m) = \int_{\Omega^-} ( \div U_n \div \overline{U_m} -  \tilde c\kappa^2U_n \overline{U_m}) $$
    $$
        = \int_{ \partial \Omega^-} \overline{U_m} \div U_n d \vec{S} - \int_{\Omega^{-}} \underbrace{( \tilde c\kappa^2 U_n + \Delta U_n)}_{=0} \overline{U_m} d \vec{S} 
    $$
    $$
        = \delta_{nm}\int_{ \partial \Omega^-} \overline{U_n} \div U_n d \vec{S}
    $$
    $$
        = \delta_{nm}|v_n|^2 \sqrt{ \tilde c} \kappa  J_n(\sqrt{ \tilde c}\kappa ) J_n'(\sqrt{ \tilde c}\kappa )  \int_{\partial \Omega^-} e^{in\phi}  e^{-in\phi} dS
    $$
    $$
        = \delta_{nm}2\pi |v_n|^2\kappa \sqrt{ \tilde c} J_n(\sqrt{ \tilde c} \kappa ) J_n'(\sqrt{ \tilde c} \kappa )
    $$
\end{proof}

\begin{remark}
    In the previous proof $U_n$ is integrated over $\Omega^-$, while $U_n$ is technically only defined on $\Gamma$. 
    However, note that, knowing the coefficients of $U_n$, it is straightforward to extend $U$ to $\Omega^-$ using our basis vectors $V_n$. 
    Also, the end result only depends on the restriction to $\Gamma$.
    \footnote{These two points is what justified the restriction of $U_n$ to $\Gamma$ in the first place.}
\end{remark}

\begin{lemma}
    $b(p_n, p_m) =  (p_n, p_m)_{H^1(\Gamma)} = \frac{(1 + n^2)}{\tilde \kappa^2 + n^2}$
\end{lemma}
\begin{proof}
    $p_n\frac{\tilde \kappa^2 + n^2}{1 + n^2}$ is orthogonal w.r.t $(-, -)_{H^1(\Gamma)}$.
\end{proof}
   
Now we can proof theorem \ref{def:galerkin_matrix}.
\begin{proof}[Proof of theorem \ref{def:galerkin_matrix}]
    Both sides of the problem in definition \ref{def:galerkin_matrix} are linear in $(V, \varphi, q)$. 
    Therefore the condition in the problem is satisfied for all 
    $(V, \varphi, q)\in \mathcal{S}^{\frac{1}{2}}_N  \times  \mathcal{S}^{-\frac{1}{2}}_N \times  \mathcal{S}^{1}_N$
    if and only it is satisfied for $(V, \varphi, q) \in (U_n, \theta_n, p_n) \forall n \in \{-N, -N+1, ..., N\}$.
    So we end up with $(2N + 1)$ systems of equations of the form 
    $$
    \begin{aligned}
        \mathrm{q}_{\kappa}(U, U_n)+\left(\mathrm{W}_{\kappa}\left(\gamma_{D}^{-} U\right), U_n\right)_{\Gamma}-\left((\frac{1}{2} \mathrm{ld}-\mathrm{K}_{\kappa}^{\prime})(\theta),U_n\right)_{\Gamma} &=\mathrm{f}_{2}(U_n) \\
        \left((\frac{1}{2} \mathrm{ld}-\mathrm{K}_{\kappa})\left(\gamma_{D}^{-} U\right), \theta_n\right)_{\Gamma}+\left(\mathrm{V}_{\kappa}(\theta), \theta_n\right)_{\Gamma}+i \overline{\eta}(p, \theta_n)_{\Gamma} &=\overline{g_2(V)}(\theta_n) \\
        -\left(\mathrm{W}_{\kappa}\left(\gamma_{D}^{-} U\right), p_n\right)_{\Gamma}-\left((\mathrm{K}_{\kappa}^{\prime}+\frac{1}{2} \mathrm{Id})(\theta), p_n\right)_{\Gamma}+\mathrm{b}(p, p_n) &=g_3(V)(p_n).
    \end{aligned}.
    $$
    Now, before plugging in for $(U, \theta, p)$ note the following structural feature: 
    Each of the summands of the left side of the problem has a factor that contains one of the following three bilinear forms:
    \begin{itemize}
        \item A scalar product $(a,b)_\Gamma$ of one of the entries of $(U, \theta, p)$ and a basis function. 
        \item The bilinear form $b(a,b)$ of of the entries of $(U, \theta, p)$ and a basis function. 
        \item The bilinear form $q_\kappa(a,b)$ of of the entries of $(U, \theta, p)$ and a basis function. 
    \end{itemize}
    Note that for each of these bilinear forms $s$ we have the orthogonality property $s(a_n, b_m) = 0$ for $n \neq m$, $a, b \in \{U, \theta, p\}$.
    Also considering that our basis are also eigenvectors of the BIO, this implies that for each of the $(2N + 1)$ problems all but one $x_n$ is in the kernel of the left hand side.
    Or, said more simply, the problem can be written as a blockdiagonal problem with $3 \times 3$ blocks. \\
    Plugging the relations derived in the former lemmata for $q_\kappa (U, U_n), b(p, p_n)$, 
    the BIOs and the scalar products elementary calculation yields
    $A_n x_n$ for the left hand side of the problem.  \\
    For the right, plugging in our extension for $f_1, f_2$ yields $b_n$.
    This concludes the proof. 
\end{proof} 

%Inwiefern anders von dem was P. Meury gemacht hat? 
% -> wir summieren nicht ueber die Nullstellen von Besselfunktionen 

\section{Validation}


\subsection{Solving a special case}
To validate the correctness of our derived matrix we show that it yields the correct numerical solution for a simple example. \\
Consider the special case
$$
\vec{f} = 
\begin{bmatrix}
    H_n^{(1)}(\kappa) -  J_n(\sqrt{\tilde c} \kappa ) \\
    H_n^{\prime (1)}(\kappa) - J_n^{\prime}(\sqrt{\tilde c} \kappa )\\
\end{bmatrix} e^{i n \phi}
$$
where $n =-N, -N+1, ..., N$ \\
Then the solution is 
$$
u = J_n(\sqrt{\tilde c} \kappa r) e^{i n \phi}, x \in \Omega^-, u = H_n^{(1)}(\kappa r)e^{i n \phi}, x \in \Omega^+.
$$
This can be seen directly by plugging in.\\
\begin{remark}
Note that the discretized problem will yield the same solution as the original one 
    as the restricted space contains the analytical solution. 
\end{remark}
\begin{proposition}
    \label{prop:anasol}
    With regards to our chosen basis, the analytical solution can be written as $(U, \theta, p) = (\frac{1}{v_n} U_n, \frac{1}{w_n} \kappa H_n^{\prime(1)}(\kappa)\theta_n, 0)$. \\
    So the solution vector should be
    $$
        C_j^U = \delta_{nj}  \frac{1}{v_n}, C_j^\theta = \delta_{nj}  \frac{1}{w_n} \kappa H_n^{\prime(1)}(\kappa), C_j^p = 0, \forall j.
    $$
\end{proposition}
\begin{proof}
    \textit{still to be written out, but the main point is:} $\theta = \partial_r(u_{\Omega^+})_{|\Gamma} = - \frac{1}{w_0} \kappa_o H_1^{(1)}(\kappa_o) \theta_0 $ and $p = 0$, (from page 33 Thesis P. Meury).
\end{proof}
Let's validate whether we get the same numerical solution using our matrix $A_n$. We just use the $\vec{b}_n$ as computed above,
with $f_1^j =  \delta_{nj} 2\pi (J_n(\sqrt{\tilde c} \kappa) - H_n^{(1)}(\kappa)), 
f_2^j = \delta_{nj} 2\pi (\sqrt{\tilde c} \kappa J_1^{\prime}(\sqrt{\tilde c}\kappa) - \kappa H_n^{\prime(1)}(\kappa))$.\\
\\
To measure how good the solution is, let's introduce the $\zeta$-number: 
\begin{definition}[ $\zeta$-number]
    For a fixed $n$ and a fixed $\kappa$, let $\vec{C}^{num}_n(\kappa)$ be the numerical solution vector to the problem \\
    $diag(A_{-N}, ..., A_N) \vec{C}^{num}_n(\kappa) = (\vec{b}_{-N}, ... , \vec{b}_{N})$. 
    Let $\vec{C}^{ana}_n(\kappa)$ be the analytical solution vector for the same problem \\
    constructed in proposition \ref{prop:anasol}.
    The $\zeta$-number is defined as 
    $$
        \zeta(\kappa)=  \max\limits_{n \in [-N, ..., N]}\frac{\|\vec{C}^{num}_n(\kappa) - \vec{C}^{ana}_n(\kappa)\|}{\|\vec{C}^{ana}_n(\kappa)\|}
    $$ 
\end{definition}
As seen in fig. \ref{fig:sol_validation}, the $\zeta$-number is negligible across all values of $\kappa$ and $n$ which validates our derivation of $A_n$ and $b_n$.
\begin{figure}
    \label{fig:sol_validation}
    \includegraphics[width=0.5\textwidth]{scenario1SolVal.pdf}
    \includegraphics[width=0.5\textwidth]{scenario2SolVal.pdf}
    \caption{Maximum relative residuum of the numerical solution by wave number. 
    On the left plot we have $\tilde c = 1/3$ (f.e.$c_i = 1, c_o = 3$)
    and on the right plot we have $\tilde c = 3/1$.
    We always use $N = 100$ from here on.}
\end{figure}
\subsection{p = 0}
We validate our matrix $A$ with another method, using the following remark from the doctoral
thesis from P.Meury:
\begin{remark}
    \label{rem:pzero}
At second glance, we realise that \(p=0\), if \((U, \theta)\) solve the problem. \\
\end{remark}
Let's assume $A_n$ is invertible\footnote{This is fair to assume for most frequencies since the variational problem is supposed to have a unique solution.}.
We can use the remark to validate that our derived matrix $A_n$ is correct, using the following proposition:
\begin{proposition}
    Let $V_{\vec{b}} := span( \vec{x}_1, \vec{x}_2)$ and let
     $P_{V_{\vec{b}}}$ be the projector onto $V_{\vec{b}}$.
     Let $P_3$ be the projector onto $(0, 0, 1)$.
     Then $P_3A_n^{-1}P_{V_{\vec{b}}} = 0$.
\end{proposition}
\begin{proof}
    Because of the remark \ref{rem:pzero}, every solution of  $A_n \vec{x} = \vec{b}_n$ satisfies $x_3 = 0$
    where $\vec{b} \in V_{\vec{b}}$. \\
    Therefore $ V_{\vec{b}} \subset Ker(P_3A_n^{-1})$. 
    Since $\vec{x}_1, \vec{x}_2$ are linearly independent, $\mathrm{dim} V_{\vec{b}} =2$, which implies $ V_{\vec{b}} = Ker(P_3A_n^{-1})$.
    So $P_3A_n^{-1}P_{V_{\vec{b}}} = 0$.
\end{proof}
    

Let's check whether this is actually satisfied 
by calculating the euclidian matrix norm of $A$ for a range of $\kappa$ values. 

As we see in fig \ref{fig:p_validation}, this is satisfied which is another validation of our matrix $A$.

\begin{figure}
    \includegraphics[width=0.5\textwidth]{scenario1PVal.pdf}
    \includegraphics[width=0.5\textwidth]{scenario2PVal.pdf}
    \caption{Euclidian matrix norm of composed matrix  $P_3A_n^{-1}P_{V_{\vec{b}}}$. 
     On the left plot we have $\tilde c = 1/3$ (f.e.$c_i = 1, c_o = 3$)
    and on the right plot we have $\tilde c = 3/1$.
    }
   \label{fig:p_validation}
\end{figure}

\section{Constructing the Solution Operator}
For better comparability, we construct the discretized solution operator for the considered problem.

Plugging the Fourier ansatz into eq. \ref{eq:helmholtz} and imposing convergence at the origin and 
the Sommerfeld radiation condition results in
\begin{align}
    u^- = \sum\limits_{n = -\infty}^\infty u_n^- \frac{J_n(\sqrt{ c_i} \tilde \kappa r)}{J_n(\sqrt{ c_i} \tilde \kappa)} e^{i n \phi} \nonumber \\
    u^+ = \sum\limits_{n = -\infty}^\infty u_n^+ \frac{H_n(\sqrt{ c_o} \tilde \kappa r)}{H_n(\sqrt{ c_o} \tilde \kappa)} e^{i n \phi} \nonumber
\end{align}
where $u_n^-$ and $u_n^+$ are the restrictions of $u$ to $\Omega^-$ und $\Omega^+$.\\
Extend $f_i = \sum\limits_{n = -\infty}^\infty f_i^n e^{in\phi}$ for $i=1,2$. The transmission condition $\gamma_{C}^{+} u^{+}=\gamma_{C}^{-} u^{-}+\mathbf{f}$ implies 
\begin{align}
    \begin{pmatrix}
        1 & -1 \\
        \sqrt{c_i} \tilde \kappa \frac{J_n^\prime(\sqrt{c_i} \tilde \kappa)}{J_n(\sqrt{c_i} \tilde \kappa)} & 
        -  \sqrt{c_o} \tilde \kappa \frac{H_n^\prime(\sqrt{c_o} \tilde \kappa)}{H_n(\sqrt{c_o} \tilde \kappa)}
    \end{pmatrix} 
    \begin{pmatrix}
        u_n^- \\u_n^+
    \end{pmatrix}
    = 
    \begin{pmatrix}
        f_n^1 \\ f_n^2
    \end{pmatrix}.
\end{align} 
By using the well-known inverse of a 2x2 matrix we obtain 
\begin{equation}
    u_n^- = \zeta((-\sqrt{c_o}\tilde \kappa J_n(\sqrt{c_i} \tilde \kappa) H_n^\prime(\sqrt{c_i} \tilde \kappa))f_n^1 
    + J_n(\sqrt{c_i} \tilde \kappa) H_n(\sqrt{c_o} \tilde \kappa) f_n^2).
\end{equation}
where $\zeta := \frac{1}{\sqrt{c_i} \tilde \kappa J_n^\prime(\sqrt{c_i} \tilde \kappa) H_n(\sqrt{c_o} \tilde \kappa)
-\sqrt{c_o} \tilde \kappa H_n^\prime(\sqrt{c_o}\tilde \kappa) J_n(\sqrt{c_i} \tilde \kappa)
}$.
This presents the solution operator, as we get $\gamma_D^-$ from restriction to the boundary.
$\gamma_N^-$ can be directly obtained by restriction of the normal derivative of $u_n^-$ to the boundary 
which is equivalent to multiplying the Fourier coefficients by $\sqrt{c_i \tilde \kappa}\frac{J_n^\prime(\sqrt{c_i}\tilde \kappa)}{J_n(\sqrt{c_i}\tilde \kappa)}$.
After rescaling the $u_n$, $f_i^n$ to the complete orthonormal system defined in eq. \ref{eq:coefficients}
we obtain the solution operator matrix: 
\begin{equation}
    S_{io}^n = 
    \zeta
    \begin{pmatrix}
        -\sqrt{c_o}\tilde \kappa J_n(\sqrt{c_i} \tilde \kappa) H_n^\prime(\sqrt{c_i} \tilde \kappa) & 
        \sqrt{n^2 + \tilde \kappa^2} J_n(\sqrt{c_i} \tilde \kappa) H_n(\sqrt{c_o} \tilde \kappa)  \\
        -\frac{1}{\sqrt{n^2 + \tilde \kappa^2}}\sqrt{c_o}\sqrt{c_i} \tilde \kappa^2 J^\prime_n(\sqrt{c_i} \tilde \kappa) H_n^\prime(\sqrt{c_i} \tilde \kappa) & 
        \sqrt{c_i} \tilde \kappa J^\prime_n(\sqrt{c_i} \tilde \kappa) H_n(\sqrt{c_o} \tilde \kappa) \\
    \end{pmatrix}
\end{equation}
that maps the Fourier coefficients of $\vec{f}$ in the complete orthonormal system for $\mathcal{H}^{\frac{1}{2}}\times \mathcal{H}^{-\frac{1}{2}}$
(as defined in eq. \ref{eq:coefficients}) to the Fouier coefficients of $\gamma_C^-u$ in the same basis.

\section{Numerical Results}
Now let's investigate maximal singular value of $A = diag(A_{-N}, A_{-N + 1}, ..., A_{N})$. 
Again, we look at the scenarios where $c_i = 1, c_o = 3$ and vice versa (fig. \ref{fig:max_sing_val})
We see that the peaks often coincide with the zeros of the Bessel function. 

\begin{figure}
    \includegraphics[width=0.5\textwidth]{scenario1MaximumSingVal.pdf}
    \includegraphics[width=0.5\textwidth]{scenario2MaximumSingVal.pdf}
    \caption{Maximum singular value of the matrix $diag(A_n)$ by wave number $\kappa$. 
     On the left plot we have $\tilde c = 1/3$ (f.e.$c_i = 1, c_o = 3$)
    and on the right plot we have $\tilde c = 3/1$. } The vertical dashed lines correspond to zeros of the Bessel functions. 
   \label{fig:max_sing_val}
\end{figure}
We can also consider the minimum singular value (fig. \ref{fig:min_sing_val}).
\begin{figure}
    \includegraphics[width=0.5\textwidth]{scenario1MinimumSingVal.pdf}
    \includegraphics[width=0.5\textwidth]{scenario2MinimumSingVal.pdf}
    \caption{Maximum singular value of the matrix $diag(A_n)$ by wave number $\kappa$. 
     On the left plot we have $\tilde c = 1/3$ (f.e.$c_i = 1, c_o = 3$)
    and on the right plot we have $\tilde c = 3/1$.  Again, the vertical dashed lines correspond to zeros of the Bessel functions. }
   \label{fig:min_sing_val}
\end{figure}
Finally, let's consider the ratio of the minimum and maximum singular value (fig. \ref{fig:ratio_sing_val}).
\begin{figure}
    \includegraphics[width=0.5\textwidth]{scenario1Ratio.pdf}
    \includegraphics[width=0.5\textwidth]{scenario2Ratio.pdf}
    \caption{Ratio of Maximum and Minimum singular value of the matrix $diag(A_n)$ by wave number $\kappa$. 
     On the left plot we have $\tilde c = 1/3$ (f.e.$c_i = 1, c_o = 3$)
    and on the right plot we have $\tilde c = 3/1$.  The vertical dashed lines correspond to zeros of the Bessel functions. }
   \label{fig:ratio_sing_val}
\end{figure}
We see that in the case $c_i = 1, c_o =3$, where the solution operator has weaker resonances resonances, 
the resonances from our matrix operator or also less strong 
than in the case $c_i = 3, c_o =1$. \cite{hiptmair2021spurious}

Use word quasi-resonances (f.e. used in Hiptmair) for peaks of sol operator 
Remark \(1.5 .\) The physical reason for the existence of quasi-resonances when \(n_{i}>n_{o}\) is that, in
this case, geometric-optic rays can undergo total internal reflection when hitting \(\Gamma\) from \(\Omega^{-}\). Rays
"hugging" the boundary via a large number of bounces with total internal reflection correspond
to solutions of the transmission problem localised near \(\Gamma\); in the asymptotic-analysis literature
these solutions are known as "whispering gallery" modes; see, e.g., \([3,4]\). The existence of quasi-
resonances of the transmission problem has only been rigorously established when \(\Omega^{-}\)is smooth
and convex with strictly positive curvature. The understanding above via rays suggests that such
quasi-resonances and quasimodes do not exist for polyhedral \(\Omega^{-}\)(since sharp corners prevent rays
from moving parallel to the boundary), although solutions with localisation qualitatively similar to
that of quasimodes can be seen when \(\Omega^{-}\)is a pentagon \([20\), Figure 13\(]\) or a hexagon \([6\), Figure 23\(] .\)

Interpretation: da $sigma_min$ stabil ist Verfahren stabil 

\section{Appendix}
resonances roots besselfunctions (Plots mit Roots in Anhang)

Weitere Werte von $c_i, c_o, nu$ testen um zu sehen ob aehnliches Ergebnis 

\bibliography{references}
\bibliographystyle{ieeetr}

\section*{Symbols}
In this section we provide a definitions of all symbols used that are not defined in the text below. 

\begin{comment}
    \subsubsection{Lipschitz Domain}
    %Another source : https://cloudflare-ipfs.com/ipfs/bafykbzacedngw4fznskrifjbxzdjkgixzzu7mwmb4sqvsywfkq3a7mdyjclfs?filename=William%20McLean%20-%20Strongly%20Elliptic%20Systems%20and%20Boundary%20Integral%20Equations%20%20-Cambridge%20University%20Press%20%282000%29.pdf 
    %Wikipedia 
    
    
    \begin{remark}
    Henceforth we shall require that, roughly speaking that \(\Omega\)  is locally the set of points located above the graph of some Lipschitz function and the boundary is this graph. 
    \end{remark}
    
    
    
    \begin{definition}[Lipschitz domain]
    Let \(n \in \mathbb{N} .\) Let \(\Omega\) be a domain of \(\mathbb{R}^{n}\) and let \(\partial \Omega\) denote the boundary of \(\Omega .\) Then \(\Omega\) is called a Lipschitz domain if for every point \(p \in \partial \Omega\) there exists a hyperplane \(H\) of dimension \(n-1\) through \(p\), a Lipschitz-continuous function \(g: H \rightarrow \mathbb{R}\) over that hyperplane, and reals \(r>0\) and \(h>0\) such that
    \begin{itemize}
        \item \(\Omega \cap C=\left\{x+y \vec{n} \mid x \in B_{r}(p) \cap H,-h<y<g(x)\right\}\)
        \item \((\partial \Omega) \cap C=\left\{x+y \vec{n} \mid x \in B_{r}(p) \cap H, g(x)=y\right\}\)
    \end{itemize}
    where \(\vec{n}\) is a unit vector that is normal to \(H\) and \(C:=\left\{x+y \vec{n} \mid x \in B_{r}(p) \cap H,-h<y<h\right\}\).
    \end{definition}
    
    
    \subsubsection{Sobolev Space}
    \begin{definition}[$H^1$]
    For a bounded domain \(\Omega \subset \mathbb{R}^{d}, d \in \mathbb{N}\), we define the Sobolev space \(H^{1}(\Omega):=\left\{v \in L^{2}(\Omega): \int_{\Omega}|\operatorname{grad} v(x)|^{2} \mathrm{~d} x<\infty\right\}\) as a Hilbert space with norm \(\|v\|_{H^{1}(\Omega)}^{2}:=\|v\|_{L^{2}(\Omega)}^{2}+|v|_{H^{1}(\Omega)}^{2}, \quad|v|_{H^{1}(\Omega)}^{2}:=\int_{\Omega}|\operatorname{grad} v(x)|^{2} \mathrm{~d} x\)
    \end{definition}
    
    
    
    
    
    
    \begin{definition}[$H^{1/2}$]
    
    \end{definition}
    
    \begin{definition}[$H^k_{loc}$]
    
    \end{definition}
    
    \subsubsection{Trace operators}
    \begin{definition}[Trace operator]
    A trace operator is a linear mapping from a function space on the volume domain \(\Omega\) to a function space on (parts of) the boundary \(\partial \Omega .\)
    \end{definition}
    
    
    \begin{definition}[(Layer) potential]
    A (layer) potential is a linear mapping from a function space on \(\partial \Omega\) into a function space on the
    volume domain \(\Omega .\)
    \end{definition}
    
    
    \begin{definition}[Dirichlet Trace]
    The Dirichlet trace (operator) \(\mathrm{T}_{D}\) boils down to pointwise restriction for smooth functions:
    $$
    \left(\mathrm{T}_{D} w\right)(\boldsymbol{x}):=w(\boldsymbol{x}) \quad \forall \boldsymbol{x} \in \Gamma, \quad w \in C^{\infty}(\bar{\Omega}).
    $$
    \end{definition}
    \begin{definition}[Dirichlet trace space]
    The Dirichlet trace space \(H^{\frac{1}{2}}(\Gamma)\) is the Hilbert space obtained by completion of \(\left.C^{\infty}(\bar{\Omega})\right|_{\Gamma}\) with
    respect to the energy norm 
    $$\|\mathfrak{u}\|_{H^{\frac{1}{2}}(\Gamma)}:=\inf \left\{\|v\|_{H^{1}(\Omega)}: v \in C^{\infty}(\bar{\Omega}), \top_{D} v=\mathfrak{u}\right\},\left.\quad \mathfrak{u} \in C^{\infty}(\bar{\Omega})\right|_{\Gamma}.$$
    \end{definition}
    
    \begin{theorem}
    The Dirichlet trace \(\mathrm{T}_{D}\) according can be extended to a continuous and surjective linear
    operator \(\mathrm{T}_{D}: H^{1}(\Omega) \rightarrow H^{\frac{1}{2}}(\Gamma)\)
    \end{theorem}
    
    \begin{definition}[Neumann Trace]
    For smooth functions the Neumann trace (operator) \(\mathrm{T}_{N}\) is defined by
    $$
    \left(\mathrm{T}_{N} w\right)(\boldsymbol{x}):=\operatorname{grad} w \cdot \boldsymbol{n}(\boldsymbol{x}) \quad \forall \boldsymbol{x} \in \Gamma, w \in C^{\infty}(\bar{\Omega}).
    $$
    \end{definition}
    
    \begin{definition}[Neumann Trace Space]
    The Neumann trace space \(H^{-\frac{1}{2}}(\Gamma)\) is the Hilbert space obtained by the completion of \(C^{0}(\Gamma)\) with respect to the norm $$\|\phi\|_{H^{-\frac{1}{2}}(\Gamma)}:=\|\widetilde{\phi}\|_{\widetilde{H}^{-1}(\Omega)}$$ where  \(\widetilde{\phi}\) is the
    "extension by zero to \(\mathbb{R}^{d \text { " }}\) of \(\phi .\)
    We have the definition \(\|\rho\|_{\widetilde{H}^{-1}(\Omega)}:=|u|_{H^{1}\left(\mathbb{R}^{3}\right)} \quad\) where \(u\) solves \(\quad\left\{\begin{array}{c}-\Delta u=\widetilde{\rho} \text { in } \mathbb{R}^{3} \\ u \text { satisfies decay conditions }\end{array}\right.\). 
    \end{definition}
    
    \begin{definition}[Space of function with square-integrable Laplacian]
    We define the Hilbert space
    $$
    H(\Delta, \Omega):=\left\{v \in H^{1}(\Omega): \Delta v \in L^{2}(\Omega)\right\}
    $$
    with norm
    $$
    \|u\|_{H(\Delta, \Omega)}^{2}:=\|u\|_{H^{1}(\Omega)}^{2}+\|\Delta u\|_{L^{2}(\Omega)}^{2}, \quad u \in H(\Delta, \Omega).
    $$
    \end{definition}
    
    \begin{theorem}
    The Neumann trace \(\mathrm{T}_{N}\) can be extended to a continuous mapping \(\mathrm{T}_{N}: H(\Delta, \Omega) \rightarrow H^{-\frac{1}{2}}(\Gamma)\).
    \end{theorem}
    
    \begin{definition}[\(C_{\mathrm{comp}}^{\infty}\left(\mathbb{R}^{d}\right)\)]
    
    \end{definition}
    
    
    \subsubsection{Notations}
    \begin{itemize}
        \item Consider a bounded Lipschitz open set $\Omega^{-} \subset \mathbb{R}^{d}$, $d=2,3$.
        \item \(\Omega^{+}:=\mathbb{R}^{d} \backslash \overline{\Omega^{-}}\)
        \item \(\Gamma:=\partial \Omega^{-}=\partial \Omega^{+}\)
        \item \(\mathbf{n}\) is the unit normal vector field on \(\Gamma\) pointing from \(\Omega^{-}\)into \(\Omega^{+}\)
        \item For any \(\varphi \in L_{\text {loc }}^{2}\left(\mathbb{R}^{d}\right)\), we let \(\varphi^{-}:=\left.\varphi\right|_{\Omega^{-}}\)and \(\varphi^{+}:=\left.\varphi\right|_{\Omega^{+}}\)
        \item \(H_{\text {loc }}^{1}\left(\Omega^{\pm}, \Delta\right):=\left\{v: \chi v \in H^{1}\left(\Omega^{\pm}\right), \Delta(\chi v) \in L^{2}\left(\Omega^{\pm}\right)\right.\)for all \(\left.\chi \in C_{\text {comp }}^{\infty}\left(\mathbb{R}^{d}\right)\right\}\)
        \item Dirichlet and Neumann trace operators\footnote{The Dirichlet trace operator boils down to pointwise restriction.}: \(\gamma_{D}^{\pm}: H_{\mathrm{loc}}^{1}\left(\Omega_{\pm}\right) \rightarrow H^{1 / 2}(\Gamma) \quad\) and \(\quad \gamma_{N}^{\pm}: H_{\mathrm{loc}}^{1}\left(\Omega_{\pm}, \Delta\right) \rightarrow H^{-1 / 2}(\Gamma)\) with \(\gamma_{D}^{\pm} v:=\left.v^{\pm}\right|_{\Gamma}\) and \(\gamma_{N}^{\pm}\)such that if \(v \in H_{\mathrm{loc}}^{2}\left(\Omega_{\pm}\right)\)then \(\gamma_{N}^{\pm} v=\mathbf{n} \cdot \gamma_{D}^{\pm}(\nabla v)\)
        \item Cauchy trace: \(\gamma_{C}^{\pm}: H_{\mathrm{loc}}^{1}\left(\Omega^{\pm}, \Delta\right) \rightarrow H^{1 / 2}(\Gamma) \times H^{-1 / 2}(\Gamma)\), \(\gamma_{C}^{\pm}:=\left(\gamma_{D}^{\pm}, \gamma_{N}^{\pm}\right)\)
        \item Sommerfeld radiation condition: \(\varphi \in C^{1}\left(\mathbb{R}^{d} \backslash B_{R}\right)\), for some ball \(B_{R}:=\{|\mathbf{x}|<R\}\), and \(\kappa>0\) satisfies this condition if \(\lim _{r \rightarrow \infty} r^{\frac{d-1}{2}}\left(\frac{\partial \varphi(\mathbf{x})}{\partial r}-\mathrm{i} \kappa \varphi(\mathbf{x})\right)=0\) in all directions. We then write \(\varphi \in \operatorname{SRC}(\kappa)\)
    %\footnote{From Wikipedia: The Sommerfeld radiation condition is used to solve uniquely the Helmholtz equation. It makes sure that \"the sources must be sources, not sinks of energy. The energy which is radiated from the sources must scatter to infinity; no energy may be radiated from infinity into ... the field.\"}
    \end{itemize}
    
    \begin{theorem}[Green's first formula]
        (From Wikipedia)
        \(\int_{U}\left(\phi \nabla^{2} \psi+\nabla \phi \cdot \nabla \psi\right) \mathrm{d} U=\int_{\partial U} \phi \frac{\partial \psi}{\partial n} \mathrm{~d} S\)
    \end{theorem}
\end{comment}
    


\end{document}

OLD STuff 

\subsection{Even more special case}
In addition to the assumptions above, pick $c_o = 1$, $\eta = 1$, and $\kappa = 5.502$ (then $J_0(\kappa) = 0$).\\
Then we have $\lambda_0^{(V)} =0$, $\lambda_n^{(K)} = \lambda_n^{(K')} = 1/2$. For easier notation write $\lambda_n^{(W)} =: \nu$.
Now we have
$$
A_0 = 
\begin{pmatrix}
    \alpha_0 + \nu P^{UU}_0  & 0 & 0\\
    0 & 0  &  i P^{p \theta} \\
    -\nu P^{Up}_0  &  -P_0^{\theta p} & 1 
\end{pmatrix}.
$$
Moreover we have 
$$
b_0 = 2 \pi H_0^{(1)}(\kappa)
\begin{bmatrix}
    -\nu \overline{\tilde{v_0}} \\
    0  \\
    \nu \overline{l_0}
\end{bmatrix}
+  2 \pi (H_1^{(1)}(\kappa) - J_1^{(1)}(\kappa))
\begin{bmatrix}
    - \overline{\tilde{v_0}} \\
    0 \\
    0
\end{bmatrix}.
$$

We have 
$$
A_0^{-1} = 
\begin{pmatrix}
    (\alpha_n + \nu P^{UU}_n)^{-1} & 0 & 0 \\
    - \frac{\nu P_0^{(Up)}}{( \alpha_0 + \nu P^{UU}_0) P_0^{\theta p}} & -\frac{i}{(P_0^{\theta p})^2} & -\frac{1}{P_0^{\theta p}} \\
    0 & - \frac{i}{P^{p \theta}_0} & 0 
\end{pmatrix}.
$$
So $
x = A_0^{-1} b_0 =  ...
\begin{pmatrix}
    %TODO: VON HIER WEITER UND HERAUSFINDEN WARUM NICHT ANALYTISCHE LOESUNG RAUSKOMMT. Eventuell divergiert v_0? 
    % Andere v_n  benutzen die nicht divergieren in der Naehe von Resonanzne? Also zB auf ganzem Kreis normieren? Dann faellt eventuell auch ein J_n oben weg 


\end{pmatrix}
$

\section{Defining the problem}
\subsection{Basic Definitions}
\subsubsection{Lipschitz Domain}
%Another source : https://cloudflare-ipfs.com/ipfs/bafykbzacedngw4fznskrifjbxzdjkgixzzu7mwmb4sqvsywfkq3a7mdyjclfs?filename=William%20McLean%20-%20Strongly%20Elliptic%20Systems%20and%20Boundary%20Integral%20Equations%20%20-Cambridge%20University%20Press%20%282000%29.pdf 
%Wikipedia 


\begin{remark}
Henceforth we shall require that, roughly speaking that \(\Omega\)  is locally the set of points located above the graph of some Lipschitz function and the boundary is this graph. 
\end{remark}



\begin{definition}[Lipschitz domain]
Let \(n \in \mathbb{N} .\) Let \(\Omega\) be a domain of \(\mathbb{R}^{n}\) and let \(\partial \Omega\) denote the boundary of \(\Omega .\) Then \(\Omega\) is called a Lipschitz domain if for every point \(p \in \partial \Omega\) there exists a hyperplane \(H\) of dimension \(n-1\) through \(p\), a Lipschitz-continuous function \(g: H \rightarrow \mathbb{R}\) over that hyperplane, and reals \(r>0\) and \(h>0\) such that
\begin{itemize}
    \item \(\Omega \cap C=\left\{x+y \vec{n} \mid x \in B_{r}(p) \cap H,-h<y<g(x)\right\}\)
    \item \((\partial \Omega) \cap C=\left\{x+y \vec{n} \mid x \in B_{r}(p) \cap H, g(x)=y\right\}\)
\end{itemize}
where \(\vec{n}\) is a unit vector that is normal to \(H\) and \(C:=\left\{x+y \vec{n} \mid x \in B_{r}(p) \cap H,-h<y<h\right\}\).
\end{definition}


\subsubsection{Sobolev Space}
\begin{definition}[$H^1$]
For a bounded domain \(\Omega \subset \mathbb{R}^{d}, d \in \mathbb{N}\), we define the Sobolev space \(H^{1}(\Omega):=\left\{v \in L^{2}(\Omega): \int_{\Omega}|\operatorname{grad} v(x)|^{2} \mathrm{~d} x<\infty\right\}\) as a Hilbert space with norm \(\|v\|_{H^{1}(\Omega)}^{2}:=\|v\|_{L^{2}(\Omega)}^{2}+|v|_{H^{1}(\Omega)}^{2}, \quad|v|_{H^{1}(\Omega)}^{2}:=\int_{\Omega}|\operatorname{grad} v(x)|^{2} \mathrm{~d} x\)
\end{definition}






\begin{definition}[$H^{1/2}$]

\end{definition}

\begin{definition}[$H^k_{loc}$]

\end{definition}

\subsubsection{Trace operators}
\begin{definition}[Trace operator]
A trace operator is a linear mapping from a function space on the volume domain \(\Omega\) to a function space on (parts of) the boundary \(\partial \Omega .\)
\end{definition}


\begin{definition}[(Layer) potential]
A (layer) potential is a linear mapping from a function space on \(\partial \Omega\) into a function space on the
volume domain \(\Omega .\)
\end{definition}


\begin{definition}[Dirichlet Trace]
The Dirichlet trace (operator) \(\mathrm{T}_{D}\) boils down to pointwise restriction for smooth functions:
$$
\left(\mathrm{T}_{D} w\right)(\boldsymbol{x}):=w(\boldsymbol{x}) \quad \forall \boldsymbol{x} \in \Gamma, \quad w \in C^{\infty}(\bar{\Omega}).
$$
\end{definition}
\begin{definition}[Dirichlet trace space]
The Dirichlet trace space \(H^{\frac{1}{2}}(\Gamma)\) is the Hilbert space obtained by completion of \(\left.C^{\infty}(\bar{\Omega})\right|_{\Gamma}\) with
respect to the energy norm 
$$\|\mathfrak{u}\|_{H^{\frac{1}{2}}(\Gamma)}:=\inf \left\{\|v\|_{H^{1}(\Omega)}: v \in C^{\infty}(\bar{\Omega}), \top_{D} v=\mathfrak{u}\right\},\left.\quad \mathfrak{u} \in C^{\infty}(\bar{\Omega})\right|_{\Gamma}.$$
\end{definition}

\begin{theorem}
The Dirichlet trace \(\mathrm{T}_{D}\) according can be extended to a continuous and surjective linear
operator \(\mathrm{T}_{D}: H^{1}(\Omega) \rightarrow H^{\frac{1}{2}}(\Gamma)\)
\end{theorem}

\begin{definition}[Neumann Trace]
For smooth functions the Neumann trace (operator) \(\mathrm{T}_{N}\) is defined by
$$
\left(\mathrm{T}_{N} w\right)(\boldsymbol{x}):=\operatorname{grad} w \cdot \boldsymbol{n}(\boldsymbol{x}) \quad \forall \boldsymbol{x} \in \Gamma, w \in C^{\infty}(\bar{\Omega}).
$$
\end{definition}

\begin{definition}[Neumann Trace Space]
The Neumann trace space \(H^{-\frac{1}{2}}(\Gamma)\) is the Hilbert space obtained by the completion of \(C^{0}(\Gamma)\) with respect to the norm $$\|\phi\|_{H^{-\frac{1}{2}}(\Gamma)}:=\|\widetilde{\phi}\|_{\widetilde{H}^{-1}(\Omega)}$$ where  \(\widetilde{\phi}\) is the
"extension by zero to \(\mathbb{R}^{d \text { " }}\) of \(\phi .\)
We have the definition \(\|\rho\|_{\widetilde{H}^{-1}(\Omega)}:=|u|_{H^{1}\left(\mathbb{R}^{3}\right)} \quad\) where \(u\) solves \(\quad\left\{\begin{array}{c}-\Delta u=\widetilde{\rho} \text { in } \mathbb{R}^{3} \\ u \text { satisfies decay conditions }\end{array}\right.\). 
\end{definition}

\begin{definition}[Space of function with square-integrable Laplacian]
We define the Hilbert space
$$
H(\Delta, \Omega):=\left\{v \in H^{1}(\Omega): \Delta v \in L^{2}(\Omega)\right\}
$$
with norm
$$
\|u\|_{H(\Delta, \Omega)}^{2}:=\|u\|_{H^{1}(\Omega)}^{2}+\|\Delta u\|_{L^{2}(\Omega)}^{2}, \quad u \in H(\Delta, \Omega).
$$
\end{definition}

\begin{theorem}
The Neumann trace \(\mathrm{T}_{N}\) can be extended to a continuous mapping \(\mathrm{T}_{N}: H(\Delta, \Omega) \rightarrow H^{-\frac{1}{2}}(\Gamma)\).
\end{theorem}

\begin{definition}[\(C_{\mathrm{comp}}^{\infty}\left(\mathbb{R}^{d}\right)\)]

\end{definition}


\subsubsection{Notations}
\begin{itemize}
    \item Consider a bounded Lipschitz open set $\Omega^{-} \subset \mathbb{R}^{d}$, $d=2,3$.
    \item \(\Omega^{+}:=\mathbb{R}^{d} \backslash \overline{\Omega^{-}}\)
    \item \(\Gamma:=\partial \Omega^{-}=\partial \Omega^{+}\)
    \item \(\mathbf{n}\) is the unit normal vector field on \(\Gamma\) pointing from \(\Omega^{-}\)into \(\Omega^{+}\)
    \item For any \(\varphi \in L_{\text {loc }}^{2}\left(\mathbb{R}^{d}\right)\), we let \(\varphi^{-}:=\left.\varphi\right|_{\Omega^{-}}\)and \(\varphi^{+}:=\left.\varphi\right|_{\Omega^{+}}\)
    \item \(H_{\text {loc }}^{1}\left(\Omega^{\pm}, \Delta\right):=\left\{v: \chi v \in H^{1}\left(\Omega^{\pm}\right), \Delta(\chi v) \in L^{2}\left(\Omega^{\pm}\right)\right.\)for all \(\left.\chi \in C_{\text {comp }}^{\infty}\left(\mathbb{R}^{d}\right)\right\}\)
    \item Dirichlet and Neumann trace operators\footnote{The Dirichlet trace operator boils down to pointwise restriction.}: \(\gamma_{D}^{\pm}: H_{\mathrm{loc}}^{1}\left(\Omega_{\pm}\right) \rightarrow H^{1 / 2}(\Gamma) \quad\) and \(\quad \gamma_{N}^{\pm}: H_{\mathrm{loc}}^{1}\left(\Omega_{\pm}, \Delta\right) \rightarrow H^{-1 / 2}(\Gamma)\) with \(\gamma_{D}^{\pm} v:=\left.v^{\pm}\right|_{\Gamma}\) and \(\gamma_{N}^{\pm}\)such that if \(v \in H_{\mathrm{loc}}^{2}\left(\Omega_{\pm}\right)\)then \(\gamma_{N}^{\pm} v=\mathbf{n} \cdot \gamma_{D}^{\pm}(\nabla v)\)
    \item Cauchy trace: \(\gamma_{C}^{\pm}: H_{\mathrm{loc}}^{1}\left(\Omega^{\pm}, \Delta\right) \rightarrow H^{1 / 2}(\Gamma) \times H^{-1 / 2}(\Gamma)\), \(\gamma_{C}^{\pm}:=\left(\gamma_{D}^{\pm}, \gamma_{N}^{\pm}\right)\)
    \item Sommerfeld radiation condition: \(\varphi \in C^{1}\left(\mathbb{R}^{d} \backslash B_{R}\right)\), for some ball \(B_{R}:=\{|\mathbf{x}|<R\}\), and \(\kappa>0\) satisfies this condition if \(\lim _{r \rightarrow \infty} r^{\frac{d-1}{2}}\left(\frac{\partial \varphi(\mathbf{x})}{\partial r}-\mathrm{i} \kappa \varphi(\mathbf{x})\right)=0\) in all directions. We then write \(\varphi \in \operatorname{SRC}(\kappa)\)
\footnote{From Wikipedia: The Sommerfeld radiation condition is used to solve uniquely the Helmholtz equation. It makes sure that \"the sources must be sources, not sinks of energy. The energy which is radiated from the sources must scatter to infinity; no energy may be radiated from infinity into ... the field.\"}
\end{itemize}
\subsection{Definition of the problem}

\begin{definition}
Given \(c_i, c_o>0\) and frequency \(k>0\), the Helmholtz transmission scattering problem is
that of finding the complex amplitude \(u\) of the sound pressure, with \(u \in H_{\text {loc }}^{1}\left(\mathbb{R}^{d} \backslash \Gamma\right)\) the solution
of \(\begin{aligned}\left(\Delta+k^{2} c_i\right) u^{-} &=0 & & \text { in } \Omega^{-} \\\left(\Delta+k^{2} c_o\right) u^{+} &=0 & & \text { in } \Omega^{+} \\ \gamma_{C}^{-} u^{-} &=\gamma_{C}^{+} u^{+}+\gamma_{C}^{\pm} u^{I} & & \text { on } \Gamma \\ u^{+} & \in \operatorname{SRC}\left(k \sqrt{c_o}\right), & & \end{aligned}\)
where the incident wave\footnote{Solution if transmission material would not be there.} \(u^{I}\) is an entire solution of the homogeneous Helmholtz equation in \(\mathbb{R}^{d}\) 
$$\left(\Delta+k^{2} c_o\right) u^{I}=0 \quad \text{ in } \mathbb{R}^{d}\
$$

\end{definition}


In principle, the jump \(\gamma_{C}^{+} u^{+}-\gamma_{C}^{-} u^{-}\)of the Cauchy trace of \(u\) across \(\Gamma\) can be more general than the Cauchy trace of an incident wave. This leads to the following generic Helmholtz transmission problem.


\begin{definition}
Given positive real numbers \(k, c_i\), and \(c_o\) and \(\mathbf{f} \in H^{1 / 2}(\Gamma) \times H^{-1 / 2}(\Gamma)\), find \(u \in H_{\mathrm{loc}}^{1}\left(\mathbb{R}^{d} \backslash \Gamma\right) \cap \operatorname{SRC}\left(k \sqrt{c_o}\right)\) such that,
\(\begin{aligned}\left(\Delta+k^{2} c_i\right) u^{-} &=0 & & i n \Omega^{-} \\\left(\Delta+k^{2} c_o\right) u^{+} &=0 & & i n \Omega^{+} \\ \gamma_{C}^{-} u^{-} &=\gamma_{C}^{+} u^{+}+\mathbf{f} & & \text { on } \Gamma \end{aligned}\)
\end{definition}
\begin{lemma}
The solution of the transmission problem of Definition \(1.1\) exists and is unique.
Moreover, if \(\mathbf{f} \in H^{1}(\Gamma) \times L^{2}(\Gamma)\) then \(\gamma_{C}^{\pm} u^{\pm} \in H^{1}(\Gamma) \times L^{2}(\Gamma)\)
\end{lemma}

\section{Solution operators}
\begin{definition}
Given positive real numbers \(k, c_{i}\), and \(c_{o}\), let
$$
S\left(c_{i}, c_{o}\right) \mathbf{f}:=\gamma_{C}^{-} u
$$
where \(u \in H_{\mathrm{loc}}^{1}\left(\mathbb{R}^{d} \backslash \Gamma\right) \cap \operatorname{SRC}\left(k \sqrt{c_{o}}\right)\) is the solution of the Helmholtz transmission problem with $c_i = c_i$. According to the earlier lemmas this is well-defined.
\end{definition}



%\begin{definition}
%We introduce the abbreviations \(S_{i o}:=S\left(c_i, c_o\right) \quad\) and \(\quad S_{o i}:=S\left(c_o, c_i\right)\). 
%\end{definition}
